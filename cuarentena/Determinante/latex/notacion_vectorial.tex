\documentclass[a4paper,10pt,spanish]{article}

\usepackage[spanish]{babel}
%\usepackage[utf8]{inputenc}
\usepackage[T1]{fontenc}

\usepackage[upright]{fourier}
\usepackage{tikz}
\usetikzlibrary{matrix,arrows,decorations.pathmorphing}

\usepackage{amsmath}

\usepackage[usenames]{color}

\usepackage{listings}

\lstset{
    language=C,
    showstringspaces=false,
    basicstyle=\ttfamily,
    keywordstyle=\color{blue},
    commentstyle=\color[gray]{0.6},
    stringstyle=\color[RGB]{40,80,20}
}

\newcommand{\inlinecode}[2][C]{\colorbox{white}{\lstinline[language=#1]$#2$}}


%opening
\title{Un Poco de Notación Matemática}
\author{txemagon}

\begin{document}

\maketitle

\begin{abstract}
Vamos a poner por escrito en lenguaje matemático los ejemplos  de vectores que hemos ido haciendo hasta ahora.
\end{abstract}

\section*{Notación de Vectores y Matrices}

Decimos que un vector es algo de la forma $\vec{v} = (1, 2, 3)$ o también $ \vec{v} = 1 \vec{i} + 2 \vec{j} + 3 \vec{k} $.
Y generalizamos diciendo: $\vec{v} = (x, y, z)$ o $ \vec{v} = x \vec{i} + y \vec{j} + z \vec{k} $.

Cuando tenemos muchos vectores solemos poner los componentes así:

\begin{gather*}
\vec{x} = (x_1, x_2,  x_3) \\
\vec{y} = (y_1, y_2,  y_3) \\
\vec{z} = (z_1, z_2,  z_3)
\end{gather*}

Pero el señor Einstein inventó una notación que nos es muy
adecuada para programar. Para él un vector quedaba representado
por un elemento genérico $ x_i $ donde $i$ toma todos los valores
de $1$ a $3$.

Así que para Einstein un vector es algo así:

$$ x_i \qquad \forall i=1 ... 3 $$

Donde $ \forall$ se lee: \emph{para todo}.

Y que nosotros escribimos muy fácilmente en programación.

\begin{lstlisting}
 for (int i=0; i<3; i++)
    a[i]
\end{lstlisting}

Y su tipo de datos: \inlinecode{double a[3];}

Decimos que la dimensión de este vector es: $3$

Las matrices tienen una forma más o menos así:
$$
\begin{pmatrix}
 1 & 2 & 3\\
 4 & 5 & 6
\end{pmatrix}
$$

Esta es una matriz de dimensión $2 \times 3$ que quiere
decir que tiene $2$ filas y $3$ columnas.

La matriz $ \mathcal{A} $ genérica sería algo así:

$$
\mathcal{A} = 
\begin{pmatrix}
 a_{11} & a_{12} & a_{13} & \dots  & a_{1n} \\
 a_{21} & a_{22} & a_{23} & \dots  & a_{2n} \\
 \vdots &        &        & \ddots & \\
 a_{m1} & a_{m2} & \dots  &        & a_{mn} 
\end{pmatrix}
$$

que el señor Einstein en su timepo representara como $a_{ij}$
que para el representaba cualquier elemento genérico de la matriz.

Un elemento genérico representa a toda la matriz pues basta con concretar
un valor para $i$ y otro para $j$ para obtener todos y cada uno de los valores.

Presentamos un cuadro resumen de las dos notaciones.

\begin{table}[h]
\centering
\begin{tabular}{r|cc}
         & Euclides        & Einstein \\ \hline
  \emph{Vector} & $ \vec{v}     $ & $ v_i $  \\
  \emph{Matriz} & $ \mathcal{A} $ & $ a_{ij} $
\end{tabular}
\end{table}

\section*{Operaciones}

\subsection*{Suma de Vectores}

Sean los vectores $ \vec{v} $ y $ \vec{w} $.

$$
\begin{align*}
\vec{v} &= (v_1, v_2, v_3) \\
\vec{w} &= (w_1, w_2, w_3) \\
\vec{z} = \vec{v} + \vec{w} &= (v_1+w_1, v_2+w_2, v_3+w_3) 
\end{align*}
$$

En notación de Einstein:

$$ z_i = v_i + w_i $$

Y en c:

\begin{lstlisting}
 for (int i=0; i<3; i++)
    z[i] = v[i] + w[i]
\end{lstlisting}


Que, como se ve, es una notación muy conveniente para un programador.



\subsection*{Producto Escalar}

Sean los vectores $ \vec{v} $ y $ \vec{w} $.

$$
\begin{align*}
\vec{v} &= (v_1, v_2, v_3) \\
\vec{w} &= (w_1, w_2, w_3) \\
p = \vec{v} \cdot \vec{w} &= v_1w_1 + v_2w_2 + v_3w_3
\end{align*}
$$

En la notación tradicional ponemos:

$$ p = v_1w_1 + v_2w_2 + v_3w_3 = \sum_{i=1}^{3} v_iw_i $$

El símbolo $ \sum $ se llama sumatorio y quiere decir que se le van dando 
valores a i y se van sumando a otros valores.

Einstein diría:

$$ p = v_iw_i $$

Y se dice que hay una contracción en $i$. Como $p$ no tiene subíndice $i$ hay 
que ir dándole todos los valores y acumulando.

En programación hacemos:

\begin{lstlisting}
 for (int i=0; i<3; i++)
    z += v[i] * w[i]
    
\end{lstlisting}

\subsection*{Determinante}

Sea la matriz $ \mathcal{A} $. Su determinante es la suma de los productos de sus diagonales principales
menos la resta de sus secundarias.

$$
det A = \left| A \right| = 
\begin{vmatrix}
 a_{11} & a_{12} & a_{13} \\
 a_{21} & a_{22} & a_{23} \\
 a_{31} & a_{32} & a_{33} 
\end{vmatrix}
$$

\begin{multline*}
 \left| A \right| = a_{11}a_{22}a_{33} + a_{21}a_{32}a_{13} + a_{31}a_{12}a_{23} - \\
 - \left( a_{13}a_{22}a_{31} + a_{23}a_{32}a_{11} + a_{33}a_{12}a_{21} \right)
\end{multline*}

Que si hacemos el truco de que cuando salimos por abajo entramos por arriba:
\begin{multline*}
 \left| A \right| = a_{11}a_{22}a_{33} + a_{21}a_{32}a_{43} + a_{31}a_{42}a_{53} - \\
 - \left( a_{13}a_{22}a_{31} + a_{23}a_{32}a_{41} + a_{33}a_{42}a_{51} \right)
\end{multline*}

\subsection*{Suma de la diagonal}

$$ S = a_{11} + a_{22} + a_{33} $$

Einstein:

$$ S = a_{ii} $$

En C:

\begin{lstlisting}
 for (int i=0; i<3; i++)
    s += a[i][i]
    
\end{lstlisting}

\subsection*{Multiplicación de Matrices}

Sea la multiplicación: $ \mathcal{C} =  \mathcal{A} \times \mathcal{B}$

El elemento $ c_{23} $ se obtine multiplicando la fila 2 de $\mathcal{A}$ por la columna 3 de $ \mathcal{B} $.

%% Author : Alain Matthes
% Source : http://altermundus.com/pages/examples.html
\documentclass[]{article}

\usepackage[utf8]{inputenc}
\usepackage[upright]{fourier}
\usepackage{tikz}
\usetikzlibrary{matrix,arrows,decorations.pathmorphing}
\begin{document}

% l' unite
\newcommand{\myunit}{1 cm}
\tikzset{
    node style sp/.style={draw,circle,minimum size=\myunit},
    node style ge/.style={circle,minimum size=\myunit},
    arrow style mul/.style={draw,sloped,midway,fill=white},
    arrow style plus/.style={midway,sloped,fill=white},
}

\begin{tikzpicture}[>=latex]
% les matrices
\matrix (A) [matrix of math nodes,%
             nodes = {node style ge},%
             left delimiter  = (,%
             right delimiter = )] at (0,0)
{%
  a_{11} & a_{12} & \ldots & a_{1p}  \\
  \node[node style sp] {a_{21}};%
         & \node[node style sp] {a_{22}};%
                  & \ldots%
                           & \node[node style sp] {a_{2p}}; \\
  \vdots & \vdots & \ddots & \vdots  \\
  a_{n1} & a_{n2} & \ldots & a_{np}  \\
};
\node [draw,below=10pt] at (A.south) 
    { $A$ : \textcolor{red}{$n$ rows} $p$ columns};

\matrix (B) [matrix of math nodes,%
             nodes = {node style ge},%
             left delimiter  = (,%
             right delimiter =)] at (6*\myunit,6*\myunit)
{%
  b_{11} & \node[node style sp] {b_{12}};%
                  & \ldots & b_{1q}  \\
  b_{21} & \node[node style sp] {b_{22}};%
                  & \ldots & b_{2q}  \\
  \vdots & \vdots & \ddots & \vdots  \\
  b_{p1} & \node[node style sp] {b_{p2}};%
                  & \ldots & b_{pq}  \\
};
\node [draw,above=10pt] at (B.north) 
    { $B$ : $p$ rows \textcolor{red}{$q$ columns}};
% matrice résultat
\matrix (C) [matrix of math nodes,%
             nodes = {node style ge},%
             left delimiter  = (,%
             right delimiter = )] at (6*\myunit,0)
{%
  c_{11} & c_{12} & \ldots & c_{1q} \\
  c_{21} & \node[node style sp,red] {c_{22}};%
                  & \ldots & c_{2q} \\
  \vdots & \vdots & \ddots & \vdots \\
  c_{n1} & c_{n2} & \ldots & c_{nq} \\
};
% les fleches
\draw[blue] (A-2-1.north) -- (C-2-2.north);
\draw[blue] (A-2-1.south) -- (C-2-2.south);
\draw[blue] (B-1-2.west)  -- (C-2-2.west);
\draw[blue] (B-1-2.east)  -- (C-2-2.east);
\draw[<->,red](A-2-1) to[in=180,out=90]
	node[arrow style mul] (x) {$a_{21}\times b_{12}$} (B-1-2);
\draw[<->,red](A-2-2) to[in=180,out=90]
	node[arrow style mul] (y) {$a_{22}\times b_{22}$} (B-2-2);
\draw[<->,red](A-2-4) to[in=180,out=90]
	node[arrow style mul] (z) {$a_{2p}\times b_{p2}$} (B-4-2);
\draw[red,->] (x) to node[arrow style plus] {$+$} (y)%
                  to node[arrow style plus] {$+\raisebox{.5ex}{\ldots}+$} (z)%
                  to (C-2-2.north west);


\node [draw,below=10pt] at (C.south) 
    {$ C=A\times B$ : \textcolor{red}{$n$ rows}  \textcolor{red}{$q$ columns}};

\end{tikzpicture}

\begin{tikzpicture}[>=latex]
% unit
% defintion of matrices
\matrix (A) [matrix of math nodes,%
             nodes = {node style ge},%
             left delimiter  = (,%
             right delimiter = )] at (0,0)
{%
  a_{11} &\ldots & a_{1k} & \ldots & a_{1p}  \\
    \vdots & \ddots & \vdots & \vdots & \vdots \\
  \node[node style sp] {a_{i1}};& \ldots%
         & \node[node style sp] {a_{ik}};%
                  & \ldots%
                           & \node[node style sp] {a_{ip}}; \\
  \vdots & \vdots& \vdots & \ddots & \vdots  \\
  a_{n1}& \ldots & a_{nk} & \ldots & a_{np}  \\
};
\node [draw,below] at (A.south) { $A$ : \textcolor{red}{$n$ rows} $p$ columns};
\matrix (B) [matrix of math nodes,%
             nodes = {node style ge},%
             left delimiter  = (,%
             right delimiter =)] at (7*\myunit,7*\myunit)
{%
  b_{11} &  \ldots& \node[node style sp] {b_{1j}};%
                  & \ldots & b_{1q}  \\
  \vdots& \ddots & \vdots & \vdots & \vdots \\
  b_{k1} &  \ldots& \node[node style sp] {b_{kj}};%
                  & \ldots & b_{kq}  \\
  \vdots& \vdots & \vdots & \ddots & \vdots \\
  b_{p1} &  \ldots& \node[node style sp] {b_{pj}};%
                  & \ldots & b_{pq}  \\
};
\node [draw,above] at (B.north) { $B$ : $p$ rows \textcolor{red}{$q$ columns}};
% matrice resultat
\matrix (C) [matrix of math nodes,%
             nodes = {node style ge},%
             left delimiter  = (,%
             right delimiter = )] at (7*\myunit,0)
{%
  c_{11} & \ldots& c_{1j} & \ldots & c_{1q} \\
  \vdots& \ddots & \vdots & \vdots & \vdots \\
    c_{i1}& \ldots & \node[node style sp,red] {c_{ij}};%
                  & \ldots & c_{iq} \\
  \vdots& \vdots & \vdots & \ddots & \vdots \\
  c_{n1}& \ldots & c_{nk} & \ldots & c_{nq} \\
};
\node [draw,below] at (C.south) 
    {$ C=A\times B$ : \textcolor{red}{$n$ rows}  \textcolor{red}{$q$ columns}};
% arrows
\draw[blue] (A-3-1.north) -- (C-3-3.north);
\draw[blue] (A-3-1.south) -- (C-3-3.south);
\draw[blue] (B-1-3.west)  -- (C-3-3.west);
\draw[blue] (B-1-3.east)  -- (C-3-3.east);
\draw[<->,red](A-3-1) to[in=180,out=90] 
    node[arrow style mul] (x) {$a_{i1}\times b_{1j}$} (B-1-3);
\draw[<->,red](A-3-3) to[in=180,out=90] 
    node[arrow style mul] (y) {$a_{ik}\times b_{kj}$}(B-3-3);
\draw[<->,red](A-3-5) to[in=180,out=90] 
    node[arrow style mul] (z) {$a_{ip}\times b_{pj}$}(B-5-3);
\draw[red,->] (x) to node[arrow style plus] {$+\raisebox{.5ex}{\ldots}+$} (y)%
                  to node[arrow style plus] {$+\raisebox{.5ex}{\ldots}+$} (z);
                  %
                  % to (C-3-3.north west);
\draw[->,red,decorate,decoration=zigzag] (z) -- (C-3-3.north west);
\end{tikzpicture}
\end{document}

% encoding : utf8
% format   : pdfLaTeX
% author   : Alain Matthes


$$ 
 c_{13} = a_{11}b_{13} + a_{12}b_{23}  + a_{13}b_{33}
$$

O como Einstein diría.

$$
c_{ij} = a_{ik}\cdot b_{kj}
$$



\end{document}